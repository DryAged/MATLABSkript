\documentclass[12pt, a4paper, twoside]{article}

\usepackage{graphicx}
\usepackage[a4paper, left=2cm, right=2cm, top=2.5cm, bottom=3cm]{geometry}

\begin{document}
    \thispagestyle{empty}
     \vspace*{4cm}
    \begin{center}
        \includegraphics[width=200pt]{Bilder/Helmut Schmidt Universität-a89973ff.png}\\
        \vspace{2cm}
        \huge\textbf{MATLAB - Grundlagen für Ingenieurwissenschaften}
    \end{center}
    \newpage

    \pagenumbering{arabic}
    \renewcommand{\contentsname}{Inhaltsverzeichnis}
    \tableofcontents
    \newpage
    \section{Einführung}
        \subsection{Was ist MATLAB}
        MATLAB ist die Abkürzung für MATrix LABoratory. Zudem ist es ein interaktives, integriertes System zur Berechnung, Visualisierung oder Programmierung mathematischer Problemstellungen. Es bietet eine einfache Skriptsprache welche auf die Verarbeitung von Matrizen ausgelegt ist.
        \subsection{Anwendungsgebiete in den Ingenieurwissenschaften}
        MATLAB bietet in vielen Ingenieurwissenschaftlichen Betätigungsfeldern weitreichende \\Vorteile.
        \begin{itemize}
            \item Signalverarbeitung
            \item Regelungstechnik
            \item FEM-Simulation
            \item Schaltungsanalyse
            \item Bildverarbeitung
            \item Datenanalyse
        \end{itemize}
        \subsection{Die Benutzeroberfläche} 
    \section{Grundlegende Operationen}
        \subsection{Variablendeklaration}
        \subsection{Mathematische Grundoperationen}
        \subsection{Kommentare}
        \subsection{Komplexe Zahlen}
    \section{Vektoren und Matrizen}
        \subsection{Erstellen von Vektoren und Matrizen}
        \subsection{Zugriff auf Elemente und Indizierung}
        \subsection{Matrixoperationen}
        \subsection{nützliche MATLAB Funktionen}
    \section{Programmiergrundlagen}
        \subsection{Skripte}
        \subsection{Funktionen}
        \subsection{Schleifen}
    \section{Arbeiten mit Dateien und Daten}
        \subsection{Speichern und Laden von Daten}
        \subsection{Importieren von Messdaten}
        \subsection{Analyse und Verarbeitung von Daten}
    \section{Visualisierung von Daten}
        \subsection{Einfache Diagramme}
        \subsection{Mehrere Kurven in einem Diagramm}
        \subsection{Mehrere Diagramme in einer Übersicht}
        \subsection{Grafische Anpassungen}
    \section{Anhang}
        \subsection{Dokumentation in MATLAB}
        \subsection{Übersicht wichtiger MATLAB Befehle}
\end{document}
