\documentclass[12pt, a4paper, twoside]{article}

\usepackage{graphicx}
\usepackage[a4paper, left=2.5cm, right=2.5cm, top=2.5cm, bottom=3cm]{geometry}
\usepackage[utf8]{inputenc}
\usepackage[T1]{fontenc}
\usepackage{tcolorbox}
\usepackage{adjustbox}
\usepackage{xparse}
\usepackage{float}
\usepackage{listings}
\usepackage{tabularx}
\usepackage{hyperref}
\usepackage[dvipsnames]{xcolor}
\usepackage[ngerman]{babel}

%Listingkonfiguration
\lstset{
  basicstyle=\ttfamily\small,
  breaklines=true,
  frame=none,
  language=MATLAB
}

\tcbuselibrary{listings, skins, breakable}

%Template für Codeerklärungsbox
\newtcolorbox{CodeErklaerungBox}[2][]{
  enhanced,
  breakable,
  colback=White,
  colframe=MidnightBlue,
  fonttitle=\bfseries,
  title=#2,
  sidebyside,
  sidebyside align=center,
  lefthand width=0.48\textwidth,
  righthand width=0.48\textwidth,
  #1
}
\newtcolorbox{Codelösung}[1]{%
  enhanced,
  title={#1},
  colback=white,
  colframe=MidnightBlue,
  boxrule=0.5pt,
  arc=2mm,
  top=1mm,
  bottom=1mm,
  left=1mm,
  right=1mm,
  fonttitle=\bfseries
}



\setlength{\fboxsep}{0.5pt}
\setlength{\fboxrule}{1pt}
\renewcommand\fbox{\fcolorbox{MidnightBlue}{White}}

\begin{document}
    \thispagestyle{empty}
     \vspace*{4cm}
    \begin{center}
        \includegraphics[width=200pt]{Bilder/Helmut Schmidt Universität-a89973ff.png}\\
        \vspace{2cm}
        \huge\textbf{MATLAB - Grundlagen für Ingenieurwissenschaften}
    \end{center}
    \newpage

    \pagenumbering{arabic}
    \renewcommand{\contentsname}{Inhaltsverzeichnis}
    \tableofcontents
    \newpage
    
    \include{Inhalt/01_Einführung}
    \section{Grundlegende Operationen}
        \subsection{Variablendeklaration}
                \begin{CodeErklaerungBox}{Einfache Wertzuweisung}
                \begin{lstlisting}
a = 3;
                \end{lstlisting}
                \tcblower
                Der Variable \texttt{a} wird der Wert \texttt{3} zugewiesen.
                \end{CodeErklaerungBox}
                \noindent Eine Zuweisung ohne ein Semikolon am Ende der Zeile bewirkt eine direkte Rückgabe des Variablenwertes.
                \begin{CodeErklaerungBox}{Fließkommazahl}
                \begin{lstlisting}
a = 4.5;
                \end{lstlisting}
                \tcblower
                Der Variable \texttt{a} wird der Wert \texttt{4.5} zugewiesen. Als Trennzeichen in MATLAB wird der Punkt an Stelle eines Kommas verwendet.
                \end{CodeErklaerungBox}
                \begin{CodeErklaerungBox}{Zeichenkette}
                \begin{lstlisting}
name = "Peter";
                \end{lstlisting}
                \tcblower
                Der Variable \texttt{name} wird der String \texttt{Peter} zugewiesen.
                \end{CodeErklaerungBox}
                \begin{CodeErklaerungBox}{Logischer Wert}
                \begin{lstlisting}
isValid = true;
                \end{lstlisting}
                \tcblower
                Der Variable \texttt{isValid} wird der boolsche Wert \texttt{true} zugewiesen.
                \end{CodeErklaerungBox}
                \begin{CodeErklaerungBox}{Automatische Typzuweisung}
                \begin{lstlisting}
a = pi;
                \end{lstlisting}
                \tcblower
                Der Variable \texttt{a} wird die, in MATLAB vordefinierte Variable $\pi$ zugewiesen.
                \end{CodeErklaerungBox}
                \noindent Neben \texttt{pi} gibt es weitere vordefinierte Variablen. Diesen kann zwar ebenfalls ein selbst definierter Wert zugewiesen werden, jedoch ist dies nicht empfehlenswert.
                \begin{center}
                \renewcommand{\arraystretch}{1.4}
                \begin{tabularx}{\textwidth}{|l| X| l|}
                    \hline
                    \textbf{Variable} & \textbf{Bedeutung} & \textbf{Wert} \\
                    \hline
                    \texttt{inf} & Unendlich & $\frac{1}{0}$ ergibt \texttt{inf} \\
                    \hline
                    \texttt{i} & Imaginäre Einheit & $\sqrt{-1}$ \\
                    \hline
                    \texttt{j} & Alternative imaginäre Einheit & $\sqrt{-1}$ \\
                    \hline
                    \texttt{NaN} & "Not a Number" - ungültiger Wert & $\frac{0}{0}$ ergibt \texttt{NaN} \\
                    \hline
                    \texttt{ans} & Ergebnis der letzten berechneten Zeile & z.B. \texttt{ans = 42} \\
                    \hline
                    \texttt{true}/\texttt{false} & Boolsche Werte & \texttt{1} bzw. \texttt{0} \\
                    \hline
                \end{tabularx}
            \end{center}
        \subsection{Mathematische Grundoperationen}
            \begin{CodeErklaerungBox}{Addition}
                \begin{lstlisting}
c = a + b;
                \end{lstlisting}
                \tcblower
                In der Variable \texttt{c} wird die Summe aus \texttt{a} und \texttt{b} gespeichert.
                \end{CodeErklaerungBox}
                \begin{CodeErklaerungBox}{Subtraktion}
                \begin{lstlisting}
c = a - b;
                \end{lstlisting}
                \tcblower
                In der Variable \texttt{c} wird die Differenz aus \texttt{a} und \texttt{b} gespeichert.
                \end{CodeErklaerungBox}
                \begin{CodeErklaerungBox}{Multiplikation}
                \begin{lstlisting}
c = a * b;
                \end{lstlisting}
                \tcblower
                In der Variable \texttt{c} wird das Produkt aus \texttt{a} und \texttt{b} gespeichert.
                \end{CodeErklaerungBox}
                \begin{CodeErklaerungBox}{Division}
                \begin{lstlisting}
c = a / b;
                \end{lstlisting}
                \tcblower
                In der Variable \texttt{c} wird der Quotient aus \texttt{a} und \texttt{b} gespeichert.
                \end{CodeErklaerungBox}
                \begin{CodeErklaerungBox}{Abrunden}
                \begin{lstlisting}
c = floor(a / b);
                \end{lstlisting}
                \tcblower
                In der Variable \texttt{c} wird das abgerundete Ergebnis der Division von \texttt{a} und \texttt{b} gespeichert.
                \end{CodeErklaerungBox}
                \begin{CodeErklaerungBox}{Aufrunden}
                \begin{lstlisting}
c = ceil(a / b);
                \end{lstlisting}
                \tcblower
                In der Variable \texttt{c} wird das aufgerundete Ergebnis der Division von \texttt{a} und \texttt{b} gespeichert.
                \end{CodeErklaerungBox}
                \begin{CodeErklaerungBox}{Modulo}
                \begin{lstlisting}
c = mod(a,b);
                \end{lstlisting}
                \tcblower
                In der Variable \texttt{c} wird der Rest der Division von \texttt{a} und \texttt{b} gespeichert.
                \end{CodeErklaerungBox}
                \begin{CodeErklaerungBox}{Potenzieren}
                \begin{lstlisting}
c = a ^ 2;
                \end{lstlisting}
                \tcblower
                In der Variable \texttt{c} wird das Ergebnis der zweiten Potenz von \texttt{a} gespeichert.
                \end{CodeErklaerungBox}
                \begin{CodeErklaerungBox}{Wurzeln}
                \begin{lstlisting}
c = sqrt(a);
                \end{lstlisting}
                \tcblower
                In der Variable \texttt{c} wird die Wurzel von \texttt{a} gespeichert.
                \end{CodeErklaerungBox}
                \begin{CodeErklaerungBox}{Betrag}
                \begin{lstlisting}
c = abs(-a);
                \end{lstlisting}
                \tcblower
                In der Variable \texttt{c} wird der Betrag von \texttt{-a} gespeichert.
                \end{CodeErklaerungBox}
        \subsection{Komplexe Zahlen}
        \begin{CodeErklaerungBox}{Definition der komplexen Zahl}
                \begin{lstlisting}
z = 2 + 3*i;
                \end{lstlisting}
                \tcblower
                Erzeugt die komplexe Zahl \texttt{z = 2 + 3i}. 
                \end{CodeErklaerungBox}
                \noindent Wie unter 2.1 beschrieben, kann \texttt{j} analog zu \texttt{i} verwendet werden.
                \begin{CodeErklaerungBox}{Real- und Imaginärteil}
                \begin{lstlisting}
re = real(z);
im = imag(z);
                \end{lstlisting}
                \tcblower
                \texttt{real()} gibt den Realteil von \texttt{z} zurück und \texttt{imag()} den Imaginärteil.
                \end{CodeErklaerungBox}
                \begin{CodeErklaerungBox}{Betrag}
                \begin{lstlisting}
r = abs(z);
                \end{lstlisting}
                \tcblower
                Berechnet den Betrag von \texttt{z} , also $\sqrt{Im(z)^2 + Re(z)^2}$.
                \end{CodeErklaerungBox}
                \begin{CodeErklaerungBox}{Winkel}
                \begin{lstlisting}
phi = angle(z);
                \end{lstlisting}
                \tcblower
                Gibt den Winkel von \texttt{z} im Bogenmaß zurück.
                \end{CodeErklaerungBox}
                \begin{CodeErklaerungBox}{Konjugation}
                \begin{lstlisting}
z_conj = conj(z);
                \end{lstlisting}
                \tcblower
                Gibt das konjugiert Komplexe der Variable \texttt{z} also $z^* = Re(z) - i\cdot Im(z)$ zurück.
                \end{CodeErklaerungBox}
                \begin{CodeErklaerungBox}{Darstellung in Polarform}
                \begin{lstlisting}
r = abs(z);
phi = angle(z);
z_polar = r * exp(1i*phi);
                \end{lstlisting}
                \tcblower
                Gibt die komplexe Zahl \texttt{z} in Polarform zurück. \texttt{exp(1i * phi)} steht für $e^{i \cdot \phi}$
                \end{CodeErklaerungBox}
                \newpage
        \subsection{Beispielaufgaben}
                \subsubsection*{Aufgabe 1}
                Gegeben sei die Funktion $f(x) = x^2 +4x +5$. Berechnen Sie die komplexen Nullstellen der Funktion und lassen Sie sich jeweils Betrag und Phase ausgeben.
                \begin{Codelösung}{Lösung 1}
                    \begin{lstlisting}
p = 4;
q = 5;

x1 = -p/2 + sqrt((p/2)^2 - q);
x2 = -p/2 - sqrt((p/2)^2 - q);

r1 = abs(x1);
r2 = abs(x2);

phi1 = angle(x1);
phi2 = angle(x2);
                    \end{lstlisting}
                    
                \end{Codelösung}


                \subsubsection*{Aufgabe 2}
                Eine elektrische Schaltung besteht aus einem Widerstand mit $R=10\Omega$ einer Spule mit $L=0,05H$ und einem Kondensator mit $C=100\mu F$. Die Reihenschaltung der drei Elemente wird bei einer Frequenz von $f=50Hz$ betrieben. Berechnen Sie die Gesamtimpedanz $Z$ dieser Schaltung.
                \begin{Codelösung}{Lösung 2}
                    \begin{lstlisting}
R = 10;
L = 0.05;
C = 100e-6;
f = 50;
omega = 2 * pi * f;

Z_R = R;
Z_L = 1j * omega * L;
Z_C = 1 / (1j * omega * C);

Z_Gesamt = Z_R + Z_L + Z_C;
                    \end{lstlisting}   
                \end{Codelösung}
                
    \section{Vektoren und Matrizen}
        \subsection{Erstellen von Vektoren und Matrizen}
        \begin{CodeErklaerungBox}{Zeilenvektor}
                \begin{lstlisting}
V = [1 2 3 4];
                \end{lstlisting}
                \tcblower
                Erzeugt einen Zeilenvektor mit den angegebenen Werten. Statt der Trennung durch ein Leerzeichen können ebenfalls Kommata verwendet werden.
                \end{CodeErklaerungBox}
                \begin{CodeErklaerungBox}{Spaltenvektor}
                \begin{lstlisting}
V = [1;2;3;4];
                \end{lstlisting}
                \tcblower
                Erzeugt einen Spaltenvektor mit den angegebenen Werten.
            \end{CodeErklaerungBox}
            \begin{CodeErklaerungBox}{Doppelpunktoperator I}
                \begin{lstlisting}
V = x1:x2;
                \end{lstlisting}
                \tcblower
                Erzeugt einen Zeilenvektor von \texttt{x1} bis \texttt{x2} in ganzzahligen Schritten.
            \end{CodeErklaerungBox}
            \begin{CodeErklaerungBox}{Doppelpunktoperator II}
                \begin{lstlisting}
V = x1:step:x2;
                \end{lstlisting}
                \tcblower
                Erzeugt einen Zeilenvektor von \texttt{x1} bis \texttt{x2} in konstanten Schritten von \texttt{step}.
            \end{CodeErklaerungBox}
            \begin{CodeErklaerungBox}{linspace}
                \begin{lstlisting}
V = linspace(x1,x2,n);
                \end{lstlisting}
                \tcblower
                Erzeugt einen Zeilenvektor von \texttt{x1} bis \texttt{x2} mit \texttt{n} gleichmäßig verteilten Werten.
            \end{CodeErklaerungBox}
            \begin{CodeErklaerungBox}{Matrizen}
                \begin{lstlisting}
A = [1 2 3; 4 5 6; 7 8 9];
                \end{lstlisting}
                \tcblower
                Elemente einer Zeile der Matrix werden wie bei den Vektoren mit Leerzeichen oder Komma getrennt. Ein Zeilenumbruch erfolgt durch Eingabe eines Semikolon.
            \end{CodeErklaerungBox}
            \begin{CodeErklaerungBox}{0-Matrix}
                \begin{lstlisting}
A = zeroes(m,n);
                \end{lstlisting}
                \tcblower
                Erzeugt eine 0-Matrix der Größe \texttt{m}x\texttt{n}. \texttt{ones()} funktioniert analog zu \texttt{zeroes()} nur mit einsen.
            \end{CodeErklaerungBox}
            \begin{CodeErklaerungBox}{Einheitsmatrix}
                \begin{lstlisting}
A = eye(n);
                \end{lstlisting}
                \tcblower
                Erzeugt die Einheitsmatrix der Größe \texttt{n}x\texttt{n}.
            \end{CodeErklaerungBox}
        \subsection{Zugriff auf Elemente und Indizierung}
        \begin{CodeErklaerungBox}{Einfache Indizierung}
                \begin{lstlisting}
V = [10 20 30 40];
V(2)
                \end{lstlisting}
                \tcblower
                Gibt den zweiten Wert des Vektors also \texttt{20} zurück.
                \end{CodeErklaerungBox}
                \begin{CodeErklaerungBox}{Indizierung in Matrizen}
                \begin{lstlisting}
A = [1 2 3; 4 5 6; 7 8 9];
A(2,3)
                \end{lstlisting}
                \tcblower
                Gibt den dritten Wert der zweiten Zeile also \texttt{6} zurück.
            \end{CodeErklaerungBox}
            \begin{CodeErklaerungBox}{Doppelpunktoperator}
                \begin{lstlisting}
A = [1 2 3; 4 5 6; 7 8 9];
A(:,3)
                \end{lstlisting}
                \tcblower
                Gibt alle Werte der dritten Spalte als Spaltenvektor zurück.
            \end{CodeErklaerungBox}
            \begin{CodeErklaerungBox}{End-Schlüsselwort}
                \begin{lstlisting}
A(end);
A(end-1);
A(:,end);
                \end{lstlisting}
                \tcblower
                Letztes Element\\
                Vorletztes Element\\
                Letzte Spalte
            \end{CodeErklaerungBox}
            \begin{CodeErklaerungBox}{Logische Indizierung}
                \begin{lstlisting}
A = [1 2 3; 4 5 6; 7 8 9];
A(A>5)
                \end{lstlisting}
                \tcblower
                Gibt den Spaltenvektor mit den Werten \texttt{7},\texttt{8},\texttt{6},\texttt{9} zurück. MATLAB prüft jedes Element der Matrix und gibt diejenigen zurück, die größer als \texttt{5} sind. Dabei wird die Matrix spaltenweise (spaltenweise Linearindizierung) durchlaufen.
\end{CodeErklaerungBox}
            \begin{CodeErklaerungBox}{Ändern von Werten}
                \begin{lstlisting}
V = [1 2 3 4];
V(3) = 7
                \end{lstlisting}
                \tcblower
                Ersetzt den dritten Wert des Vektor durch 7.
            \end{CodeErklaerungBox}
        \subsection{Matrixoperationen}
        \begin{CodeErklaerungBox}{Elementweise Operationen}
                \begin{lstlisting}
A = [1 2; 3 4];
B = [5 6; 7 8];

C = A + B;
D = A - B;
E = A .* B;
F = A ./ B;
G = A .^ 2;
                \end{lstlisting}
                \tcblower
                Für Elementweise Operationen muss ein Punkt vor dem Operator genutzt werden. Bei Addition und Subtraktion ist dies jedoch irrelevant.
            \end{CodeErklaerungBox}
            \begin{CodeErklaerungBox}{Matrixmultiplikation}
                \begin{lstlisting}
A = [1 2; 3 4];
B = [5; 6];

C = A * B;
                \end{lstlisting}
                \tcblower
                Für die klassische Matrixmultiplikation gelten die allgemeinen Regeln aus der linearen Algebra. Beim Operator muss hier auf den Punkt verzichtet werden.
            \end{CodeErklaerungBox}
            \begin{CodeErklaerungBox}{Transposition}
                \begin{lstlisting}
A = [1 2; 3 4];
B = A'
                \end{lstlisting}
                \tcblower
                Für Matrizen mit komplexen Zahlen erzeugt das Hochkomma die adjungierte Matrix. Für reine Transposition wird \texttt{.'} verwendet. Bei reelen Matrizen können beide Versionen analog verwendet werden.
            \end{CodeErklaerungBox}
            \begin{CodeErklaerungBox}{Lösen linearer Gleichungssysteme der Form {Ax = b}}
                \begin{lstlisting}
A = [1 2; 3 4];
b = [5;6];
x = A \ b;
                \end{lstlisting}
                \tcblower
                Zum Lösen von linearen Gleichungssystemen wird der Backslash verwendet.
            \end{CodeErklaerungBox}
        \subsection{nützliche MATLAB Funktionen}
        \renewcommand{\arraystretch}{1.4}
        \begin{tabularx}{\textwidth}{|X|X|}
            \hline
            \textbf{Funktion} & \textbf{Rückgabe} \\
            \hline
            \texttt{max(A)} & Höchster Wert der Matrix\\
            \hline
            \texttt{min(A)} & Kleinster Wert der Matrix \\
            \hline
            \texttt{sum(A)} & Summer der einzelnen Spalten \\
            \hline
            \texttt{mean(A)} & Mittelwert der einzelnen Spalten \\
            \hline
            \texttt{length(A)} & Zeilenanzahl der Matrix \\
            \hline
            \texttt{numel(A)} & Spaltenanzahl der Matrix \\
            \hline
            \texttt{size(A)} & Zeilen- und Spaltenanzahl der Matrix \\
            \hline
            \texttt{det(A)} & Determinante der Matrix \\
            \hline
            \texttt{rank(A)} & Rang der Matrix \\
            \hline
            \texttt{trace(A)} & Summe der Diagonalelemente \\
            \hline
            \texttt{eig(A)} & Eigenverte der Matrix \\
            \hline
            \texttt{flipud(A)} & vertikale Spiegelung der Matrix \\
            \hline
            \texttt{fliplr(A)} & horizontale Spiegelung der Matrix \\
            \hline
        \end{tabularx}
        \subsection{Beispielaufgaben}
            \subsubsection*{Aufgabe 1}
            Gegeben seien folgende Matrizen:
            \begin{math}
                A = \begin{pmatrix}
                    1 & 2 \\
                    3 & 4
                \end{pmatrix}
            \end{math}
            und 
            \begin{math}
                B = \begin{pmatrix}
                    5 & 6 \\
                    7 & 8
                \end{pmatrix}
            \end{math}
            .\\

            \vspace{0.5cm}
            \noindent
            1. Berechnen Sie die Summe der beiden Matrizen \\
            2. Berechnen sie das elementweise Quadrat von A. \\
            3. Bestimmen Sie die transponierte von B.

            \begin{Codelösung}{Lösung 1}
                \begin{lstlisting}
A = [1 2; 3 4];
B = [5 6; 7 8];

C = A + B;

D = A .^ 2;

E = B';
                \end{lstlisting}
                
            \end{Codelösung}
            \subsubsection*{Aufgabe 2}

            Gegeben sei die Matrix $A = \begin{pmatrix}
                2 & 5 & 8  \\
                7 & 9 & 3  \\
                6 & 1 & 0  \\
            \end{pmatrix}$\\

            \vspace{0.5cm}
            \noindent1. Geben Sie das 2. Element der ersten Zeile aus.\\
            2. Ersetzen Sie das 2. Element der dritte Spalte durch 1\\
            3. Ermitteln Sie alle Einträge, die größer als 6 sind.

            \begin{Codelösung}{Lösung 2}
                \begin{lstlisting}
A = [2 5 8; 7 9 3; 6 1 0];         

a = A(1,2);

A(2,3) = 1;

B = A(A>6);
                \end{lstlisting}
            \end{Codelösung}
            \subsubsection*{Aufgabe 3}
            Gegeben sei folgendes lineares Gleichungssystem:
            
                \[2x + 3y - z = 1\]
                \[4x + 1y + 1z = 9\]
                \[-2x + 5y + 2z = 2\]

            \noindent 1. Lösen Sie das Gleichungssystem.\\
            2. Überprüfen Sie das Ergebnis durch Rückeinsetzen.

            \begin{Codelösung}{Lösung 3}
                \begin{lstlisting}
A = [2 3 -1; 4 1 1; -2 5 2];
b = [1; 9; 2];

x = A \ b;

test = A * x;
                \end{lstlisting}
            \end{Codelösung}
    \section{Programmiergrundlagen}
        \subsection{Skripte}
        Ein Skript ist eine Sammlung von MATLAB Befehlen, die in einer Datei gemeinsam abgespeichert und ausgeführt werden können. Die Dateiendung eines solchen Skripts ist \texttt{.m} 
        \subsubsection*{Vorgehensweise}
        \begin{itemize}
                \item Rechtsklick im Current Folder \textrightarrow New \textrightarrow Script
                \item Eingeben der gewünschten MATLAB Befehle im Editor analog zur Verwendung im Command Window
                \item Speichern des Skriptes
                \item Ausführen durch Eingabe des Dateinamens ohne Dateiendung im Command Window oder den Run Button in der Navigationsleiste bei geöffnetem Editor.
        \end{itemize}
        \subsubsection*{Kommentare}
        Mittels des \texttt{\%} Zeichens kann ein Kommentar eingefügt werden, das beim Ausführen des Skriptes nicht beachtet wird.
        \begin{Codelösung}{Skriptbeispiel}
                \begin{lstlisting}
% Erzeuge 3x3 Matrix
A = [1 2 3; 4 5 6; 7 8 9];

% Erste Spalte der Matrix als Spaltenvektor
b = A(:,1);

% Transponiert b zu Zeilenvektor
c = b';
                \end{lstlisting}
        \end{Codelösung}
        \subsubsection*{Vorteile von Skripten gegenüber der Eingabe im Command Window}
        \begin{itemize}
                \item Skripte können beliebig oft ausgeführt werden, ohne die Befehle jedes mal erneut eingeben zu müssen.
                \item Rechenweg bleibt komplett dokumentiert und kann einfacher überprüft und angepasst werden.
                \item Komplexe Abläufe lassen sich klar gliedern und durch Kommentare strukturieren.
        \end{itemize}
        \clearpage
        \subsection{Funktionen}
        Funktionen werden ebenfalls in einem \texttt{.m}-File gespeichert enthalten im Gegensatz zu einem einfachen Skript jedoch Ein- und Ausgabeargumente und werden mit einem \texttt{end} beendet.
        \begin{center}
                \texttt{function [Ausgabe] = Funktionsname(Eingabe)}     
        \end{center}
                Die Verwendung von eckigen Klammern ist lediglich bei Verwendung mehrerer Ausgabeargumente notwendig, kann jedoch auch bei nur einem Argument verwendet werden. \\
                Der Funktionsname sollte dem Dateinamen entsprechen, um die Funktion auch in anderen Skripten ausführen zu können.
                \begin{Codelösung}{Funktionsbeispiel}
                        \begin{lstlisting}
function A = Flaecheninhalt(r)

A = pi * r^2;
end       
                        \end{lstlisting}

                \end{Codelösung}
                \noindent
                Die Funktion kann nun durch Eingabe im Command Window oder innerhalb eines Skriptes mit beispielsweise
                \begin{center}
                        \texttt{Flaecheninhalt(3)}
                \end{center}
                aufgerufen werden und gibt somit den Flächeninhalt eines Kreises mit dem Radius 3 zurück.
        \subsection{Schleifen}
     \section{Arbeiten mit Dateien und Daten}
        \subsection{Speichern und Laden von Daten}
        \begin{CodeErklaerungBox}{alle Variablen speichern}
              \begin{lstlisting}
save('dateiname.mat');
              \end{lstlisting}
              \tcblower
              Speichert alle Variablen des Workspaces in einer \texttt{.mat} Datei.
        \end{CodeErklaerungBox}
        \begin{CodeErklaerungBox}{einzelne Variablen speichern}
              \begin{lstlisting}
save('dateiname.mat' 
, 'variable');
              \end{lstlisting}
              \tcblower
              Speichert eine ausgewählte Variable in einer \texttt{.mat} Datei.
        \end{CodeErklaerungBox}
                \begin{CodeErklaerungBox}{alle Variablen laden}
              \begin{lstlisting}
load('dateiname.mat');
              \end{lstlisting}
              \tcblower
              Lädt alle Variablen einer \texttt{.mat} Datei in den Workspace.
        \end{CodeErklaerungBox}
        \begin{CodeErklaerungBox}{einzelne Variablen laden}
              \begin{lstlisting}
load('dateiname.mat' 
, 'variable');
              \end{lstlisting}
              \tcblower
              Lädt die ausgewählte Variable einer \texttt{.mat} Datei in den Workspace.
        \end{CodeErklaerungBox}
       \noindent Alternativ zum Laden mittels \texttt{load()} kann dies auch über einen Doppelklick auf die \texttt{.mat} Datei im Current Folder ausgeführt werden.
       \begin{CodeErklaerungBox}{Speichern als .csv Datei}
              \begin{lstlisting}
dlmwrite('dateiname.csv' , Matrixname);
              \end{lstlisting}
              \tcblower
              Speichert eine ausgewählte Matrix als \texttt{.csv} Datei.
        \end{CodeErklaerungBox}
        \begin{CodeErklaerungBox}{Speichern als ASCII}
              \begin{lstlisting}
save('dateiname.txt','variable','-ascii')
              \end{lstlisting}
              \tcblower
              Speichert eine ausgewählte Variable im ASCII-Format in einer Textdatei. (Zum Austausch mit anderen Systemen)
        \end{CodeErklaerungBox}
        \subsection{Importieren von Messdaten}
        \subsection{Analyse und Verarbeitung von Daten}
        
     \section{Visualisierung von Daten}
        \subsection{Einfache Diagramme}
        \begin{CodeErklaerungBox}{Plotten von Kurven}
                \begin{lstlisting}
V = [1 2 3 4];
plot(V);
                \end{lstlisting}
                \tcblower
                Plottet die angegebenen Werte in einem Diagramm und verbindet die Werte mit einer Graden.
                \end{CodeErklaerungBox}
                 \begin{CodeErklaerungBox}{Plotten von Datenpunkten}
                \begin{lstlisting}
V = [1 2 3 4];
plot(V,'o');
                \end{lstlisting}
                \tcblower
                Plottet die angegebenen Werte als dedizierte Punkte im Diagramm.
                \end{CodeErklaerungBox}
        \subsection{Mehrere Kurven in einem Diagramm}
        \subsection{Mehrere Diagramme in einer Übersicht}
        \subsection{Grafische Anpassungen}
        \subsection{Mehrere Kurven in einem Diagramm}
        \subsection{Mehrere Diagramme in einer Übersicht}
        \subsection{Grafische Anpassungen}
       
     \section{Anhang}
        \subsection{Dokumentation in MATLAB}
        \subsection{Übersicht wichtiger MATLAB Befehle}
    
    
   
   
   
\end{document}
