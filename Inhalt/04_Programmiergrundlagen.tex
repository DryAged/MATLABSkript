\section{Programmiergrundlagen}
        \subsection{Skripte}
        Ein Skript ist eine Sammlung von MATLAB Befehlen, die in einer Datei gemeinsam abgespeichert und ausgeführt werden können. Die Dateiendung eines solchen Skripts ist \texttt{.m} 
        \subsubsection*{Vorgehensweise}
        \begin{itemize}
                \item Rechtsklick im Current Folder \textrightarrow New \textrightarrow Script
                \item Eingeben der gewünschten MATLAB Befehle im Editor analog zur Verwendung im Command Window
                \item Speichern des Skriptes
                \item Ausführen durch Eingabe des Dateinamens ohne Dateiendung im Command Window oder den Run Button in der Navigationsleiste bei geöffnetem Editor.
        \end{itemize}
        \subsubsection*{Kommentare}
        Mittels des \texttt{\%} Zeichens kann ein Kommentar eingefügt werden, das beim Ausführen des Skriptes nicht beachtet wird.
        \begin{Codelösung}{Beispielskript}
                \begin{lstlisting}
% Erzeuge 3x3 Matrix
A = [1 2 3; 4 5 6; 7 8 9];

% Erste Spalte der Matrix als Spaltenvektor
b = A(:,1);

% Transponiert b zu Zeilenvektor
c = b';
                \end{lstlisting}
        \end{Codelösung}
        \subsubsection*{Vorteile von Skripten gegenüber der Eingabe im Command Window}
        \begin{itemize}
                \item Skripte können beliebig oft ausgeführt werden, ohne die Befehle jedes mal erneut eingeben zu müssen.
                \item Rechenweg bleibt komplett dokumentiert und kann einfacher überprüft und angepasst werden.
                \item Komplexe Abläufe lassen sich klar gliedern und durch Kommentare strukturieren.
        \end{itemize}
        \subsection{Funktionen}
        \subsection{Schleifen}