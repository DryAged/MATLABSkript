 \section{Arbeiten mit Dateien und Daten}
        \subsection{Speichern und Laden von Daten}
        \begin{CodeErklaerungBox}{alle Variablen speichern}
              \begin{lstlisting}
save('dateiname.mat');
              \end{lstlisting}
              \tcblower
              Speichert alle Variablen des Workspaces in einer \texttt{.mat} Datei.
        \end{CodeErklaerungBox}
        \begin{CodeErklaerungBox}{einzelne Variablen speichern}
              \begin{lstlisting}
save('dateiname.mat' 
, 'variable');
              \end{lstlisting}
              \tcblower
              Speichert eine ausgewählte Variable in einer \texttt{.mat} Datei.
        \end{CodeErklaerungBox}
                \begin{CodeErklaerungBox}{alle Variablen laden}
              \begin{lstlisting}
load('dateiname.mat');
              \end{lstlisting}
              \tcblower
              Lädt alle Variablen einer \texttt{.mat} Datei in den Workspace.
        \end{CodeErklaerungBox}
        \begin{CodeErklaerungBox}{einzelne Variablen laden}
              \begin{lstlisting}
load('dateiname.mat' 
, 'variable');
              \end{lstlisting}
              \tcblower
              Lädt die ausgewählte Variable einer \texttt{.mat} Datei in den Workspace.
        \end{CodeErklaerungBox}
       \noindent Alternativ zum Laden mittels \texttt{load()} kann dies auch über einen Doppelklick auf die \texttt{.mat} Datei im Current Folder ausgeführt werden.
       \begin{CodeErklaerungBox}{Speichern als .csv Datei}
              \begin{lstlisting}
dlmwrite('dateiname.csv' , Matrixname);
              \end{lstlisting}
              \tcblower
              Speichert eine ausgewählte Matrix als \texttt{.csv} Datei.
        \end{CodeErklaerungBox}
        \begin{CodeErklaerungBox}{Speichern als ASCII}
              \begin{lstlisting}
save('dateiname.txt','variable','-ascii');
              \end{lstlisting}
              \tcblower
              Speichert eine ausgewählte Variable im ASCII-Format in einer Textdatei. (Zum Austausch mit anderen Systemen)
        \end{CodeErklaerungBox}
        \subsection{Importieren von Messdaten}
        \begin{CodeErklaerungBox}{Import von Tabellen}
            \begin{lstlisting}
readtable('dateiname');
            \end{lstlisting}
            \tcblower
            Importiert eine Tabelle mit Headerzeilen und gibt ein \texttt{table} - Objekt zurück.
        \end{CodeErklaerungBox}
        \begin{CodeErklaerungBox}{Import von Matrizen}
            \begin{lstlisting}
readmatrix('dateiname');
            \end{lstlisting}
            \tcblower
            Importiert eine rein numerische Tabelle, ignoriert Headerzeilen und gibt eine \texttt{double} - Matrix zurück.
        \end{CodeErklaerungBox}
        \subsection{Analyse und Verarbeitung von Daten}
        \begin{CodeErklaerungBox}{Mittelwert}
            \begin{lstlisting}
mean(daten);
            \end{lstlisting}
            \tcblower
            Gibt den Spaltenweisen Mittelwert einer Datenmatrix oder den Mittelwert eines Spaltenvektors zurück.
        \end{CodeErklaerungBox}
        \begin{CodeErklaerungBox}{Standardabweichung}
            \begin{lstlisting}
std(daten);
            \end{lstlisting}
            \tcblower
            Gibt die spaltenweise Standardabweichung einer Datenmatrix oder die Standardabweichung eines Datenvektors zurück.
        \end{CodeErklaerungBox}
        