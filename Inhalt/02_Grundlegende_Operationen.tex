\section{Grundlegende Operationen}
        \subsection{Variablendeklaration}
                \begin{CodeErklaerungBox}{Einfache Wertzuweisung}
                \begin{lstlisting}
a = 3;
                \end{lstlisting}
                \tcblower
                Der Variable \texttt{a} wird der Wert \texttt{3} zugewiesen.
                \end{CodeErklaerungBox}
                \noindent Eine Zuweisung ohne ein Semikolon am Ende der Zeile bewirkt eine direkte Rückgabe des Variablenwertes.
                \begin{CodeErklaerungBox}{Fließkommazahl}
                \begin{lstlisting}
a = 4.5;
                \end{lstlisting}
                \tcblower
                Der Variable \texttt{a} wird der Wert \texttt{4.5} zugewiesen. Als Trennzeichen in MATLAB wird der Punkt an Stelle eines Kommas verwendet.
                \end{CodeErklaerungBox}
                \begin{CodeErklaerungBox}{Zeichenkette}
                \begin{lstlisting}
name = "Peter";
                \end{lstlisting}
                \tcblower
                Der Variable \texttt{name} wird der String \texttt{Peter} zugewiesen.
                \end{CodeErklaerungBox}
                \begin{CodeErklaerungBox}{Logischer Wert}
                \begin{lstlisting}
isValid = true;
                \end{lstlisting}
                \tcblower
                Der Variable \texttt{isValid} wird der boolsche Wert \texttt{true} zugewiesen.
                \end{CodeErklaerungBox}
                \begin{CodeErklaerungBox}{Automatische Typzuweisung}
                \begin{lstlisting}
a = pi;
                \end{lstlisting}
                \tcblower
                Der Variable \texttt{a} wird die, in MATLAB vordefinierte Variable $\pi$ zugewiesen.
                \end{CodeErklaerungBox}
                \noindent Neben \texttt{pi} gibt es weitere vordefinierte Variablen. Diesen kann zwar ebenfalls ein selbst definierter Wert zugewiesen werden, jedoch ist dies nicht empfehlenswert.
                \begin{center}
                \renewcommand{\arraystretch}{1.4}
                \begin{tabularx}{\textwidth}{|l| X| l|}
                    \hline
                    \textbf{Variable} & \textbf{Bedeutung} & \textbf{Wert} \\
                    \hline
                    \texttt{inf} & Unendlich & $\frac{1}{0}$ ergibt \texttt{inf} \\
                    \hline
                    \texttt{i} & Imaginäre Einheit & $\sqrt{-1}$ \\
                    \hline
                    \texttt{j} & Alternative imaginäre Einheit & $\sqrt{-1}$ \\
                    \hline
                    \texttt{NaN} & "Not a Number" - ungültiger Wert & $\frac{0}{0}$ ergibt \texttt{NaN} \\
                    \hline
                    \texttt{ans} & Ergebnis der letzten berechneten Zeile & z.B. \texttt{ans = 42} \\
                    \hline
                    \texttt{true}/\texttt{false} & Boolsche Werte & \texttt{1} bzw. \texttt{0} \\
                    \hline
                \end{tabularx}
            \end{center}
        \subsection{Mathematische Grundoperationen}
            \begin{CodeErklaerungBox}{Addition}
                \begin{lstlisting}
c = a + b;
                \end{lstlisting}
                \tcblower
                In der Variable \texttt{c} wird die Summe aus \texttt{a} und \texttt{b} gespeichert.
                \end{CodeErklaerungBox}
                \begin{CodeErklaerungBox}{Subtraktion}
                \begin{lstlisting}
c = a - b;
                \end{lstlisting}
                \tcblower
                In der Variable \texttt{c} wird die Differenz aus \texttt{a} und \texttt{b} gespeichert.
                \end{CodeErklaerungBox}
                \begin{CodeErklaerungBox}{Multiplikation}
                \begin{lstlisting}
c = a * b;
                \end{lstlisting}
                \tcblower
                In der Variable \texttt{c} wird das Produkt aus \texttt{a} und \texttt{b} gespeichert.
                \end{CodeErklaerungBox}
                \begin{CodeErklaerungBox}{Division}
                \begin{lstlisting}
c = a / b;
                \end{lstlisting}
                \tcblower
                In der Variable \texttt{c} wird der Quotient aus \texttt{a} und \texttt{b} gespeichert.
                \end{CodeErklaerungBox}
                \begin{CodeErklaerungBox}{Abrunden}
                \begin{lstlisting}
c = floor(a / b);
                \end{lstlisting}
                \tcblower
                In der Variable \texttt{c} wird das abgerundete Ergebnis der Division von \texttt{a} und \texttt{b} gespeichert.
                \end{CodeErklaerungBox}
                \begin{CodeErklaerungBox}{Aufrunden}
                \begin{lstlisting}
c = ceil(a / b);
                \end{lstlisting}
                \tcblower
                In der Variable \texttt{c} wird das aufgerundete Ergebnis der Division von \texttt{a} und \texttt{b} gespeichert.
                \end{CodeErklaerungBox}
                \begin{CodeErklaerungBox}{Modulo}
                \begin{lstlisting}
c = mod(a,b);
                \end{lstlisting}
                \tcblower
                In der Variable \texttt{c} wird der Rest der Division von \texttt{a} und \texttt{b} gespeichert.
                \end{CodeErklaerungBox}
                \begin{CodeErklaerungBox}{Potenzieren}
                \begin{lstlisting}
c = a ^ 2;
                \end{lstlisting}
                \tcblower
                In der Variable \texttt{c} wird das Ergebnis der zweiten Potenz von \texttt{a} gespeichert.
                \end{CodeErklaerungBox}
                \begin{CodeErklaerungBox}{Wurzeln}
                \begin{lstlisting}
c = sqrt(a);
                \end{lstlisting}
                \tcblower
                In der Variable \texttt{c} wird die Wurzel von \texttt{a} gespeichert.
                \end{CodeErklaerungBox}
                \begin{CodeErklaerungBox}{Betrag}
                \begin{lstlisting}
c = abs(-a);
                \end{lstlisting}
                \tcblower
                In der Variable \texttt{c} wird der Betrag von \texttt{-a} gespeichert.
                \end{CodeErklaerungBox}
        \subsection{Komplexe Zahlen}
        \begin{CodeErklaerungBox}{Definition der komplexen Zahl}
                \begin{lstlisting}
z = 2 + 3*i;
                \end{lstlisting}
                \tcblower
                Erzeugt die komplexe Zahl \texttt{z = 2 + 3i}. 
                \end{CodeErklaerungBox}
                \noindent Wie unter 2.1 beschrieben, kann \texttt{j} analog zu \texttt{i} verwendet werden.
                \begin{CodeErklaerungBox}{Real- und Imaginärteil}
                \begin{lstlisting}
re = real(z);
im = imag(z);
                \end{lstlisting}
                \tcblower
                \texttt{real()} gibt den Realteil von \texttt{z} zurück und \texttt{imag()} den Imaginärteil.
                \end{CodeErklaerungBox}
                \begin{CodeErklaerungBox}{Betrag}
                \begin{lstlisting}
r = abs(z);
                \end{lstlisting}
                \tcblower
                Berechnet den Betrag von \texttt{z} , also $\sqrt{Im(z)^2 + Re(z)^2}$.
                \end{CodeErklaerungBox}
                \begin{CodeErklaerungBox}{Winkel}
                \begin{lstlisting}
phi = angle(z);
                \end{lstlisting}
                \tcblower
                Gibt den Winkel von \texttt{z} im Bogenmaß zurück.
                \end{CodeErklaerungBox}
                \begin{CodeErklaerungBox}{Konjugation}
                \begin{lstlisting}
z_conj = conj(z);
                \end{lstlisting}
                \tcblower
                Gibt das konjugiert Komplexe der Variable \texttt{z} also $z^* = Re(z) - i\cdot Im(z)$ zurück.
                \end{CodeErklaerungBox}
                \begin{CodeErklaerungBox}{Darstellung in Polarform}
                \begin{lstlisting}
r = abs(z);
phi = angle(z);
z_polar = r * exp(1i*phi);
                \end{lstlisting}
                \tcblower
                Gibt die komplexe Zahl \texttt{z} in Polarform zurück. \texttt{exp(1i * phi)} steht für $e^{i \cdot \phi}$
                \end{CodeErklaerungBox}
                \newpage
        \subsection{Beispielaufgaben}
                \subsubsection*{Aufgabe 1}
                Gegeben sei die Funktion $f(x) = x^2 +4x +5$. Berechnen Sie die komplexen Nullstellen der Funktion und lassen Sie sich jeweils Betrag und Phase ausgeben.
                \begin{Codelösung}{Lösung 1}
                    \begin{lstlisting}
p = 4;
q = 5;

x1 = -p/2 + sqrt((p/2)^2 - q);
x2 = -p/2 - sqrt((p/2)^2 - q);

r1 = abs(x1);
r2 = abs(x2);

phi1 = angle(x1);
phi2 = angle(x2);
                    \end{lstlisting}
                    
                \end{Codelösung}


                \subsubsection*{Aufgabe 2}
                Eine elektrische Schaltung besteht aus einem Widerstand mit $R=10\Omega$ einer Spule mit $L=0,05H$ und einem Kondensator mit $C=100\mu F$. Die Reihenschaltung der drei Elemente wird bei einer Frequenz von $f=50Hz$ betrieben. Berechnen Sie die Gesamtimpedanz $Z$ dieser Schaltung.
                \begin{Codelösung}{Lösung 2}
                    \begin{lstlisting}
R = 10;
L = 0.05;
C = 100e-6;
f = 50;
omega = 2 * pi * f;

Z_R = R;
Z_L = 1j * omega * L;
Z_C = 1 / (1j * omega * C);

Z_Gesamt = Z_R + Z_L + Z_C;
                    \end{lstlisting}   
                \end{Codelösung}
                