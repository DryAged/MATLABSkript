\section{Vektoren und Matrizen}
        \subsection{Erstellen von Vektoren und Matrizen}
        \begin{CodeErklaerungBox}{Zeilenvektor}
                \begin{lstlisting}
v = [1 2 3 4];
                \end{lstlisting}
                \tcblower
                Erzeugt einen Zeilenvektor mit den angegebenen Werten. Statt der Trennung durch ein Leerzeichen können ebenfalls Kommata verwendet werden.
                \end{CodeErklaerungBox}
                \begin{CodeErklaerungBox}{Spaltenvektor}
                \begin{lstlisting}
v = [1;2;3;4];
                \end{lstlisting}
                \tcblower
                Erzeugt einen Spaltenvektor mit den angegebenen Werten.
            \end{CodeErklaerungBox}
            \begin{CodeErklaerungBox}{Doppelpunktoperator I}
                \begin{lstlisting}
v = x1:x2;
                \end{lstlisting}
                \tcblower
                Erzeugt einen Zeilenvektor von \texttt{x1} bis \texttt{x2} in ganzzahligen Schritten.
            \end{CodeErklaerungBox}
            \begin{CodeErklaerungBox}{Doppelpunktoperator II}
                \begin{lstlisting}
v = x1:step:x2;
                \end{lstlisting}
                \tcblower
                Erzeugt einen Zeilenvektor von \texttt{x1} bis \texttt{x2} in konstanten Schritten von \texttt{step}.
            \end{CodeErklaerungBox}
            \begin{CodeErklaerungBox}{linspace}
                \begin{lstlisting}
v = linspace(x1,x2,n);
                \end{lstlisting}
                \tcblower
                Erzeugt einen Zeilenvektor von \texttt{x1} bis \texttt{x2} mit \texttt{n} gleichmäßig verteilten Werten.
            \end{CodeErklaerungBox}
            \begin{CodeErklaerungBox}{Matrizen}
                \begin{lstlisting}
A = [1 2 3; 4 5 6; 7 8 9];
                \end{lstlisting}
                \tcblower
                Elemente einer Zeile der Matrix werden wie bei den Vektoren mit Leerzeichen oder Komma getrennt. Ein Zeilenumbruch erfolgt durch Eingabe eines Semikolon.
            \end{CodeErklaerungBox}
            \begin{CodeErklaerungBox}{0-Matrix}
                \begin{lstlisting}
A = zeroes(m,n);
                \end{lstlisting}
                \tcblower
                Erzeugt eine 0-Matrix der Größe \texttt{m}x\texttt{n}. \texttt{ones()} funktioniert analog zu \texttt{zeroes()} nur mit einsen.
            \end{CodeErklaerungBox}
            \begin{CodeErklaerungBox}{Einheitsmatrix}
                \begin{lstlisting}
A = eye(n);
                \end{lstlisting}
                \tcblower
                Erzeugt die Einheitsmatrix der Größe \texttt{n}x\texttt{n}.
            \end{CodeErklaerungBox}
        \subsection{Zugriff auf Elemente und Indizierung}
        \subsection{Matrixoperationen}
        \subsection{nützliche MATLAB Funktionen}