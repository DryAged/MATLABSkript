\section{Vektoren und Matrizen}
        \subsection{Erstellen von Vektoren und Matrizen}
        \begin{CodeErklaerungBox}{Zeilenvektor}
                \begin{lstlisting}
V = [1 2 3 4];
                \end{lstlisting}
                \tcblower
                Erzeugt einen Zeilenvektor mit den angegebenen Werten. Statt der Trennung durch ein Leerzeichen können ebenfalls Kommata verwendet werden.
                \end{CodeErklaerungBox}
                \begin{CodeErklaerungBox}{Spaltenvektor}
                \begin{lstlisting}
V = [1;2;3;4];
                \end{lstlisting}
                \tcblower
                Erzeugt einen Spaltenvektor mit den angegebenen Werten.
            \end{CodeErklaerungBox}
            \begin{CodeErklaerungBox}{Doppelpunktoperator I}
                \begin{lstlisting}
V = x1:x2;
                \end{lstlisting}
                \tcblower
                Erzeugt einen Zeilenvektor von \texttt{x1} bis \texttt{x2} in ganzzahligen Schritten.
            \end{CodeErklaerungBox}
            \begin{CodeErklaerungBox}{Doppelpunktoperator II}
                \begin{lstlisting}
V = x1:step:x2;
                \end{lstlisting}
                \tcblower
                Erzeugt einen Zeilenvektor von \texttt{x1} bis \texttt{x2} in konstanten Schritten von \texttt{step}.
            \end{CodeErklaerungBox}
            \begin{CodeErklaerungBox}{linspace}
                \begin{lstlisting}
V = linspace(x1,x2,n);
                \end{lstlisting}
                \tcblower
                Erzeugt einen Zeilenvektor von \texttt{x1} bis \texttt{x2} mit \texttt{n} gleichmäßig verteilten Werten.
            \end{CodeErklaerungBox}
            \begin{CodeErklaerungBox}{Matrizen}
                \begin{lstlisting}
A = [1 2 3; 4 5 6; 7 8 9];
                \end{lstlisting}
                \tcblower
                Elemente einer Zeile der Matrix werden wie bei den Vektoren mit Leerzeichen oder Komma getrennt. Ein Zeilenumbruch erfolgt durch Eingabe eines Semikolon.
            \end{CodeErklaerungBox}
            \begin{CodeErklaerungBox}{0-Matrix}
                \begin{lstlisting}
A = zeroes(m,n);
                \end{lstlisting}
                \tcblower
                Erzeugt eine 0-Matrix der Größe \texttt{m}x\texttt{n}. \texttt{ones()} funktioniert analog zu \texttt{zeroes()} nur mit einsen.
            \end{CodeErklaerungBox}
            \begin{CodeErklaerungBox}{Einheitsmatrix}
                \begin{lstlisting}
A = eye(n);
                \end{lstlisting}
                \tcblower
                Erzeugt die Einheitsmatrix der Größe \texttt{n}x\texttt{n}.
            \end{CodeErklaerungBox}
        \subsection{Zugriff auf Elemente und Indizierung}
        \begin{CodeErklaerungBox}{Einfache Indizierung}
                \begin{lstlisting}
V = [10 20 30 40];
V(2)
                \end{lstlisting}
                \tcblower
                Gibt den zweiten Wert des Vektors also \texttt{20} zurück.
                \end{CodeErklaerungBox}
                \begin{CodeErklaerungBox}{Indizierung in Matrizen}
                \begin{lstlisting}
A = [1 2 3; 4 5 6; 7 8 9];
A(2,3)
                \end{lstlisting}
                \tcblower
                Gibt den dritten Wert der zweiten Zeile also \texttt{6} zurück.
            \end{CodeErklaerungBox}
            \begin{CodeErklaerungBox}{Doppelpunktoperator}
                \begin{lstlisting}
A = [1 2 3; 4 5 6; 7 8 9];
A(:,3)
                \end{lstlisting}
                \tcblower
                Gibt alle Werte der dritten Spalte als Spaltenvektor zurück.
            \end{CodeErklaerungBox}
            \begin{CodeErklaerungBox}{End-Schlüsselwort}
                \begin{lstlisting}
A(end);
A(end-1);
A(:,end);
                \end{lstlisting}
                \tcblower
                Letztes Element\\
                Vorletztes Element\\
                Letzte Spalte
            \end{CodeErklaerungBox}
            \begin{CodeErklaerungBox}{Logische Indizierung}
                \begin{lstlisting}
A = [1 2 3; 4 5 6; 7 8 9];
A(A>5)
                \end{lstlisting}
                \tcblower
                Gibt den Spaltenvektor mit den Werten \texttt{7},\texttt{8},\texttt{6},\texttt{9} zurück. MATLAB prüft jedes Element der Matrix und gibt diejenigen zurück, die größer als \texttt{5} sind. Dabei wird die Matrix spaltenweise (spaltenweise Linearindizierung) durchlaufen.
\end{CodeErklaerungBox}
            \begin{CodeErklaerungBox}{Ändern von Werten}
                \begin{lstlisting}
V = [1 2 3 4];
V(3) = 7
                \end{lstlisting}
                \tcblower
                Ersetzt den dritten Wert des Vektor durch 7.
            \end{CodeErklaerungBox}
        \subsection{Matrixoperationen}
        \subsection{nützliche MATLAB Funktionen}
        \subsection{Beispielaufgaben}